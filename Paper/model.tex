% !TEX TS-program = pdflatexmk
 
\documentclass[11pt,english]{article}
\usepackage[margin=1in]{geometry}
\usepackage{amsmath,amssymb, bm, amsthm}

% FONT: either newpxtext or mathpazo. quite similar
\usepackage{newpxtext, newpxmath}
%\usepackage{mathpazo}

\usepackage{graphicx, import, subcaption, rotating, verbatim, float}
\usepackage{setspace}
\usepackage{pdflscape}
\usepackage{booktabs} 

\usepackage[table]{xcolor}
\usepackage{nicematrix}

\usepackage[authoryear]{natbib}
\bibliographystyle{ecta} 
\setlength{\bibsep}{0.0pt}
\def\bibfont{\small}
\usepackage{color} 
\definecolor{red}{RGB}{153,0,0}
\definecolor{blue}{RGB}{0,0,153}
\definecolor{green}{RGB}{0,102,0}

\usepackage[most]{tcolorbox}
\usepackage{parcolumns}  
 \usepackage[linesnumbered, ruled, vlined]{algorithm2e} 

\usepackage{enumitem}

\usepackage{tabularx, colortbl, xcolor, makecell, booktabs}
\newcolumntype{Y}{>{\centering\arraybackslash}X}
\definecolor{lightgray}{gray}{0.95}

\usepackage[page]{appendix}
 
%\usepackage{tocloft}
\usepackage{titlesec}

\usepackage{soul} %for highlight \hl{} command

%\usepackage{etoolbox}
\usepackage{etoc}

%code font
\newcommand{\code}[1]{\texttt{#1}}

\newcommand{\E}{\mathrm{E}}
\newcommand{\pc}[1]{\dot{#1}}
\newcommand{\er}[1]{(\ref{eq:#1})}
\newcommand{\fr}[1]{Figure~\ref{fig:#1}}
\newcommand{\tr}[1]{Table~\ref{tab:#1}}
\newcommand{\sr}[1]{Section~\ref{sec:#1}}
\newcommand{\ar}[1]{Appendix~\ref{app:#1}}

\newtheorem{theorem}{Theorem}
\newtheorem{assumption}{Assumption}
\newtheorem{definition}{Definition}
\newtheorem{lemma}{Lemma}
\newtheorem{proposition}{Proposition}
\newtheorem{corollary}{Corollary}
\newtheorem{example}{Example}


\makeatletter
% *, 1, 2, ...
\renewcommand*{\@fnsymbol}[1]{\ifcase#1\or \else\@arabic{\numexpr#1-1\relax}\fi}


%%%%%%%%%%%%%%%%%%%%%%%%%%%%%%%%%%%%%%%%%%
% REF AND HYPERLINK COLOURS

%\usepackage[pdftex, hidelinks]{hyperref}
%\hypersetup{colorlinks=false, linkcolor=red, filecolor=red, urlcolor=red, citecolor=red}

\usepackage{hyperref}        
\hypersetup{
    bookmarks=true,         % show bookmarks bar?
    unicode=false,          % non-Latin characters in Acrobat?s bookmarks
    pdftoolbar=true,        % show Acrobat?s toolbar?
    pdfmenubar=true,        % show Acrobat?s menu?
    pdffitwindow=false,     % window fit to page when opened
    pdfstartview={FitH},    % fits the width of the page to the window
    pdftitle={Job Mobility and Unemployment Risk},    % title
    pdfauthor={Clymo Denderski Mercan and Schoefer},     % author
%    pdfsubject={Subject},   % subject of the document
%    pdfcreator={Creator},   % creator of the document
%    pdfproducer={Producer}, % producer of the document
%    pdfkeywords={keyword1} {key2} {key3}, % list of keywords
    pdfnewwindow=true,      % links in new window
    colorlinks=true,     % color of internal links (change box color with linkbordercolor)
    citecolor=blue,        % color of links to bibliography
    filecolor=blue,      % color of file links
    urlcolor=blue,           % color of external links
    linkcolor=blue
    }
%%%%%%%%%%%%%%%%%%%%%%%%%%%%%%%%%%%%%%%%%%


\title{Continuous time incomplete market codes \thanks{\noindent \hspace*{-2em} %We are grateful to XXX. The views expressed in this paper are those of the authors and do not necessarily reflect the position of XXX. Clymo: 
Contact: \url{alex.clymo@psemail.eu}. 
}}

\author{Alex Clymo \\ PSE}

%\author{%
%\begin{tabular}{cc c}
%Alex Clymo  & \;\;\;\;\;\;\;\;\; &  Other  \\
%\textit{ PSE }  & \;\;\;\;\;\;\;\;\; &   \textit{ Institution} \\
% \; & \; \\
%Other  & \;\;\;\;\;\;\;\;\; & Other  \\
%\textit{Institution}  & \;\;\;\;\;\;\;\;\; & \textit{Institution} \\
%\end{tabular}
%}

%\date{June 2099}

\begin{document}

\maketitle


%\begin{abstract}
%\noindent Some basic codes. 
%\end{abstract}

\onehalfspacing

\vspace{-1cm}

\section{Introduction} 
\label{sec:introduction} 

This repository provides some basic but clean codes for the solution of heterogeneous agent incomplete markets models à la \cite{BewleyStationary1986}, \cite{ImrohorogluCost1989}, \cite{HuggettRiskfree1993}, and \cite{AiyagariUninsured1994} in continuous time. The codes are applications of the \cite{AchdouEtAlIncome2022} methods, and details of the methods can be found in that paper and its online appendix. 

The codes set out a generic consumer problem which can be applied to all models. A switch then allows you to control whether we solve the model in partial equilibrium or in the general equilibrium of either \cite{AiyagariUninsured1994} (i.e. with capital) or \cite{HuggettRiskfree1993} (i.e. with bonds). The problem is set and solved in steady state without aggregate shocks. 


\section{A generic consumer problem}

\subsection{Agent's problem}

A unit continuum of infinitely lived agents faces both idiosyncratic and aggregate productivity risk. Each agent saves using asset holdings \( a \geq \underline{a} \), subject to a borrowing constraint $\underline{a} \leq 0$. Agents inelastically provide one unit of labour, and have idiosyncratic productivity shocks \( z \), which follow a Poisson jump process with discrete values $z_j$ for $j=1,...,N_z$. A new value is drawn at rate $\alpha_z$ from a Markov distribution $\gamma_{j,j'}$. They take the wage $w$ and interest rate $r$ as given. 

Agents choose consumption \( c \) to maximize expected lifetime utility, discounted at rate \( \rho \). The value function \( v(a, z ) \) satisfies the Hamilton-Jacobi-Bellman (HJB) equation:
\begin{equation}
\label{eq:hjb1}
\rho v(a, z_j) = \max_{c} \, u(c) + v_a(a, z_j )\big(w z_j + r a - c\big)  + \alpha_z \left( \sum_{j'} \pi_{j,j'} v(a, z_{j'}) - v(a, z_j) \right)
\end{equation}

\subsection{Kolmogorov Forward Equation} 

$\mu(a,z)$ is the equilibrium distribution of agents across asset and productivity levels. The distribution evolves over time according to the Kolmogorov Forward (KF, or Fokker-Planck) equation. Letting $\dot \mu(a, z)$ denote the time derivative, we have
\begin{equation}
\label{eq:kf1}
\dot \mu(a, z_j) = - \frac{\partial}{\partial a} \left[ \left( w z_j + r  a - c(a,z_j) \right) \mu(a, z_j) \right]  + \alpha_z \sum_{j'} \pi_{j,j'} \mu(a, z_{j'}) - \alpha_z \mu(a, z_j)
\end{equation}
In steady state $\dot \mu(a, z_j) = 0$. 


\section{Market clearing and calibration} 

One unit of time is taken as one year. We take standard values for most parameters. We fix the borrowing constraint to be equal to the quarterly average wage: $\underline a = -1/4 w E[z]$. The code provides options for several income processes. 

\subsection{Partial equilibrium}

The worker takes some fixed $w$ and $r$ as given and we solve the value function and distribution. 

\subsection{\cite{HuggettRiskfree1993} style model}

There is no capital and production is only through labour, giving a fixed wage which we set to $w=1$. Workers save in a bond in zero net supply, so the interest rate $r$ adjusts so that $\int a \, d\mu(a,z) = 0$. We use a loop to adjust $r$ to clear the market. 

\subsection{\cite{AiyagariUninsured1994} style model}

Firms operate a constant returns production function and rent capital and labor from households. The aggregate wage and interest rate satisfy:
\begin{equation}
\label{eq:prices}
w = Z F_L(K, L), \qquad r = Z F_K(K, L) - \delta
\end{equation}
where aggregate output $Y$ is produced from the production function \( Y = Z F(K, L) \). \( K \) and \( L \) denote aggregate capital and efficiency units of labor, and \( \delta \) is the depreciation rate. Given the assumed idiosyncratic risk process, total labour is fixed at, \( L = \int z \, d\mu(a,z) = \bar L\). Market clearing requires:
\begin{equation}
K = \int a \, d\mu(a,z), \qquad C = \int c(a,z) \, d\mu(a,z)
\end{equation}
Here $c(a,z)$ is the consumption policy function in equilibrium. These conditions equivalently imply the aggregate resource constraint $Y = C + I$ where $I = \int \dot a(a,z) \, d\mu(a,z) - \delta K$ is aggregate investment. 

We choose $Z$ to target $w=1$. We use a loop to jointly find the equilibrium $K$ and calibrate $Z$. 




%\clearpage
%\section{Conclusion} 
%\label{sec:conclusion}
%
%XXX

\bibliography{HAML_refs}


 
%\clearpage
%
%% \restoregeometry
%% \FloatBarrier
%
%\appendix
%%\pagenumbering{arabic} % restarts page numbering for Appendix
%\begin{center}
%\textbf{\LARGE APPENDIX}
%\end{center}
%
%\vspace{-2em}
%
%\setcounter{figure}{0} \renewcommand{\thefigure}{A.\arabic{figure}}
%\setcounter{table}{0} \renewcommand{\thetable}{A.\arabic{table}}
%\setcounter{equation}{0} \renewcommand{\theequation}{A.\arabic{equation}}
%
%\singlespacing
%\addtocontents{lof}{\protect\setcounter{tocdepth}{1}}
%\vspace{-.2cm}
%\addtocontents{lot}{\protect\setcounter{tocdepth}{1}}
%\etocsettagdepth{mtchapter}{none}
%\etocsettagdepth{mtappendix}{subsection}
%
%
%\section{Appendix to XXX}
%
%
%


\end{document}
